\documentclass[]{article}
\usepackage{lmodern}
\usepackage{amssymb,amsmath}
\usepackage{ifxetex,ifluatex}
\usepackage{fixltx2e} % provides \textsubscript
\ifnum 0\ifxetex 1\fi\ifluatex 1\fi=0 % if pdftex
  \usepackage[T1]{fontenc}
  \usepackage[utf8]{inputenc}
\else % if luatex or xelatex
  \ifxetex
    \usepackage{mathspec}
  \else
    \usepackage{fontspec}
  \fi
  \defaultfontfeatures{Ligatures=TeX,Scale=MatchLowercase}
\fi
% use upquote if available, for straight quotes in verbatim environments
\IfFileExists{upquote.sty}{\usepackage{upquote}}{}
% use microtype if available
\IfFileExists{microtype.sty}{%
\usepackage{microtype}
\UseMicrotypeSet[protrusion]{basicmath} % disable protrusion for tt fonts
}{}
\usepackage[margin=1in]{geometry}
\usepackage{hyperref}
\hypersetup{unicode=true,
            pdftitle={Pokemon Challenge},
            pdfauthor={Chan Le},
            pdfborder={0 0 0},
            breaklinks=true}
\urlstyle{same}  % don't use monospace font for urls
\usepackage{color}
\usepackage{fancyvrb}
\newcommand{\VerbBar}{|}
\newcommand{\VERB}{\Verb[commandchars=\\\{\}]}
\DefineVerbatimEnvironment{Highlighting}{Verbatim}{commandchars=\\\{\}}
% Add ',fontsize=\small' for more characters per line
\usepackage{framed}
\definecolor{shadecolor}{RGB}{248,248,248}
\newenvironment{Shaded}{\begin{snugshade}}{\end{snugshade}}
\newcommand{\AlertTok}[1]{\textcolor[rgb]{0.94,0.16,0.16}{#1}}
\newcommand{\AnnotationTok}[1]{\textcolor[rgb]{0.56,0.35,0.01}{\textbf{\textit{#1}}}}
\newcommand{\AttributeTok}[1]{\textcolor[rgb]{0.77,0.63,0.00}{#1}}
\newcommand{\BaseNTok}[1]{\textcolor[rgb]{0.00,0.00,0.81}{#1}}
\newcommand{\BuiltInTok}[1]{#1}
\newcommand{\CharTok}[1]{\textcolor[rgb]{0.31,0.60,0.02}{#1}}
\newcommand{\CommentTok}[1]{\textcolor[rgb]{0.56,0.35,0.01}{\textit{#1}}}
\newcommand{\CommentVarTok}[1]{\textcolor[rgb]{0.56,0.35,0.01}{\textbf{\textit{#1}}}}
\newcommand{\ConstantTok}[1]{\textcolor[rgb]{0.00,0.00,0.00}{#1}}
\newcommand{\ControlFlowTok}[1]{\textcolor[rgb]{0.13,0.29,0.53}{\textbf{#1}}}
\newcommand{\DataTypeTok}[1]{\textcolor[rgb]{0.13,0.29,0.53}{#1}}
\newcommand{\DecValTok}[1]{\textcolor[rgb]{0.00,0.00,0.81}{#1}}
\newcommand{\DocumentationTok}[1]{\textcolor[rgb]{0.56,0.35,0.01}{\textbf{\textit{#1}}}}
\newcommand{\ErrorTok}[1]{\textcolor[rgb]{0.64,0.00,0.00}{\textbf{#1}}}
\newcommand{\ExtensionTok}[1]{#1}
\newcommand{\FloatTok}[1]{\textcolor[rgb]{0.00,0.00,0.81}{#1}}
\newcommand{\FunctionTok}[1]{\textcolor[rgb]{0.00,0.00,0.00}{#1}}
\newcommand{\ImportTok}[1]{#1}
\newcommand{\InformationTok}[1]{\textcolor[rgb]{0.56,0.35,0.01}{\textbf{\textit{#1}}}}
\newcommand{\KeywordTok}[1]{\textcolor[rgb]{0.13,0.29,0.53}{\textbf{#1}}}
\newcommand{\NormalTok}[1]{#1}
\newcommand{\OperatorTok}[1]{\textcolor[rgb]{0.81,0.36,0.00}{\textbf{#1}}}
\newcommand{\OtherTok}[1]{\textcolor[rgb]{0.56,0.35,0.01}{#1}}
\newcommand{\PreprocessorTok}[1]{\textcolor[rgb]{0.56,0.35,0.01}{\textit{#1}}}
\newcommand{\RegionMarkerTok}[1]{#1}
\newcommand{\SpecialCharTok}[1]{\textcolor[rgb]{0.00,0.00,0.00}{#1}}
\newcommand{\SpecialStringTok}[1]{\textcolor[rgb]{0.31,0.60,0.02}{#1}}
\newcommand{\StringTok}[1]{\textcolor[rgb]{0.31,0.60,0.02}{#1}}
\newcommand{\VariableTok}[1]{\textcolor[rgb]{0.00,0.00,0.00}{#1}}
\newcommand{\VerbatimStringTok}[1]{\textcolor[rgb]{0.31,0.60,0.02}{#1}}
\newcommand{\WarningTok}[1]{\textcolor[rgb]{0.56,0.35,0.01}{\textbf{\textit{#1}}}}
\usepackage{longtable,booktabs}
\usepackage{graphicx,grffile}
\makeatletter
\def\maxwidth{\ifdim\Gin@nat@width>\linewidth\linewidth\else\Gin@nat@width\fi}
\def\maxheight{\ifdim\Gin@nat@height>\textheight\textheight\else\Gin@nat@height\fi}
\makeatother
% Scale images if necessary, so that they will not overflow the page
% margins by default, and it is still possible to overwrite the defaults
% using explicit options in \includegraphics[width, height, ...]{}
\setkeys{Gin}{width=\maxwidth,height=\maxheight,keepaspectratio}
\IfFileExists{parskip.sty}{%
\usepackage{parskip}
}{% else
\setlength{\parindent}{0pt}
\setlength{\parskip}{6pt plus 2pt minus 1pt}
}
\setlength{\emergencystretch}{3em}  % prevent overfull lines
\providecommand{\tightlist}{%
  \setlength{\itemsep}{0pt}\setlength{\parskip}{0pt}}
\setcounter{secnumdepth}{0}
% Redefines (sub)paragraphs to behave more like sections
\ifx\paragraph\undefined\else
\let\oldparagraph\paragraph
\renewcommand{\paragraph}[1]{\oldparagraph{#1}\mbox{}}
\fi
\ifx\subparagraph\undefined\else
\let\oldsubparagraph\subparagraph
\renewcommand{\subparagraph}[1]{\oldsubparagraph{#1}\mbox{}}
\fi

%%% Use protect on footnotes to avoid problems with footnotes in titles
\let\rmarkdownfootnote\footnote%
\def\footnote{\protect\rmarkdownfootnote}

%%% Change title format to be more compact
\usepackage{titling}

% Create subtitle command for use in maketitle
\providecommand{\subtitle}[1]{
  \posttitle{
    \begin{center}\large#1\end{center}
    }
}

\setlength{\droptitle}{-2em}

  \title{Pokemon Challenge}
    \pretitle{\vspace{\droptitle}\centering\huge}
  \posttitle{\par}
    \author{Chan Le}
    \preauthor{\centering\large\emph}
  \postauthor{\par}
      \predate{\centering\large\emph}
  \postdate{\par}
    \date{2/26/2020}

\usepackage{booktabs}
\usepackage{longtable}
\usepackage{array}
\usepackage{multirow}
\usepackage{wrapfig}
\usepackage{float}
\usepackage{colortbl}
\usepackage{pdflscape}
\usepackage{tabu}
\usepackage{threeparttable}
\usepackage{threeparttablex}
\usepackage[normalem]{ulem}
\usepackage{makecell}
\usepackage{xcolor}

\begin{document}
\maketitle

\hypertarget{i.-introduction}{%
\section{\texorpdfstring{\textbf{I.
Introduction}}{I. Introduction}}\label{i.-introduction}}

\hypertarget{a.-the-pocket-monsters-universe}{%
\subsection{\texorpdfstring{\textbf{A. The Pocket Monsters
Universe}}{A. The Pocket Monsters Universe}}\label{a.-the-pocket-monsters-universe}}

Pokemon (Pocket Monsters) is orginally a series of console video games
developed by GameFreak, published by Nintendo, and is later adapted into
numerous different media. Since its original work of Pokemon Red and
Blue, Pokeon has grown into one of the most well-known media franchise
of all time.

The Pokemon universe is based on the fitional creatures of the same name
and their relationship with human. In the Pokemon world, Trainers
capture wild Pokemons with Pokeballs and raise them as their own
partners. Trainers can also battle among each others to gain experience,
and have one common goal of defeating the Elite Four and the Pokemon
Champion, who are said to be the strongest Pokemon Trainers yet.

Up until Generation 6, there are 721 Pokemon species founded in the
wild. Some of them have multiple forms (which differ in the distribution
of stats), raising the number of all different versions of Pokemon
species captured in this data set to .

\hypertarget{b.-base-stat-values---pokemon-species-uniqueness}{%
\subsection{\texorpdfstring{\textbf{B. Base Stat Values - Pokemon
Species
Uniqueness}}{B. Base Stat Values - Pokemon Species Uniqueness}}\label{b.-base-stat-values---pokemon-species-uniqueness}}

Principally, a Pokemon is determined by its statistics (stats). There
are six main permanant stats that make up a Pokemon's over strength:

\begin{itemize}
\tightlist
\item
  HP (Hit Points): determine how much damage a Pokémon can receive
  before fainting.
\item
  Atk (Attack): partly determines how much damage a Pokémon deals when
  using a physical move.
\item
  Def (Defense): partly determines how much damage a Pokémon receives
  when it is hit with a physical move.
\item
  SpAtk (Special Attack): partly determines how much damage a Pokémon
  deals when using a special move.
\item
  SpDef (Special Defense): partly determines how much damage a Pokémon
  receives when it is hit with a special move.
\item
  Spe (Speed): determines the order of Pokémon that can act in battle.
\end{itemize}

There are many elements that can affect these stats, so that hardly no
two Pokemons are identical. However, each Pokemon species will have its
own set of base stat values, which in turn has a great impact on a
certain individual Pokemon at any level. This table shows the first 5
Pokemon (species) and their base stats:

\begin{Shaded}
\begin{Highlighting}[]
  \KeywordTok{library}\NormalTok{(tidyverse)}
\end{Highlighting}
\end{Shaded}

\begin{verbatim}
## -- Attaching packages --------------------------------------------------------------------------- tidyverse 1.3.0 --
\end{verbatim}

\begin{verbatim}
## <U+2713> ggplot2 3.2.1     <U+2713> purrr   0.3.3
## <U+2713> tibble  2.1.3     <U+2713> dplyr   0.8.3
## <U+2713> tidyr   1.0.0     <U+2713> stringr 1.4.0
## <U+2713> readr   1.3.1     <U+2713> forcats 0.4.0
\end{verbatim}

\begin{verbatim}
## -- Conflicts ------------------------------------------------------------------------------ tidyverse_conflicts() --
## x dplyr::filter() masks stats::filter()
## x dplyr::lag()    masks stats::lag()
\end{verbatim}

\begin{Shaded}
\begin{Highlighting}[]
\CommentTok{# # Input data and reset name ----}
\NormalTok{data <-}\StringTok{ }\KeywordTok{read.csv2}\NormalTok{(}\StringTok{"pokemon.csv"}\NormalTok{, }\DataTypeTok{sep =} \StringTok{","}\NormalTok{)}

\KeywordTok{colnames}\NormalTok{(data) <-}\StringTok{ }\KeywordTok{c}\NormalTok{(}\StringTok{"index"}\NormalTok{,}\StringTok{"name"}\NormalTok{,}\StringTok{"type1"}\NormalTok{,}\StringTok{"type2"}\NormalTok{,}\StringTok{"hp"}\NormalTok{,}\StringTok{"atk"}\NormalTok{,}\StringTok{"def"}\NormalTok{,}\StringTok{"spAtk"}\NormalTok{,}\StringTok{"spDef"}\NormalTok{,}\StringTok{"spe"}\NormalTok{,}\StringTok{"gen"}\NormalTok{,}\StringTok{"legendary"}\NormalTok{)}

\NormalTok{statName <-}\StringTok{ }\KeywordTok{c}\NormalTok{(}\StringTok{"hp"}\NormalTok{,}\StringTok{"atk"}\NormalTok{,}\StringTok{"def"}\NormalTok{,}\StringTok{"spAtk"}\NormalTok{,}\StringTok{"spDef"}\NormalTok{,}\StringTok{"spe"}\NormalTok{)}

\NormalTok{  data }\OperatorTok\StringTok{ }\KeywordTok{head}\NormalTok{(}\DecValTok{5}\NormalTok{) }\OperatorTok\StringTok{ }\KeywordTok{select}\NormalTok{(name, hp, atk, def, spAtk, spDef, spe) }\OperatorTok\StringTok{ }
\StringTok{  }\NormalTok{knitr}\OperatorTok{::}\KeywordTok{kable}\NormalTok{(, }\DataTypeTok{col.names =} \KeywordTok{c}\NormalTok{(}\StringTok{"Name"}\NormalTok{,}\StringTok{"HP"}\NormalTok{,}\StringTok{"Attack"}\NormalTok{,}\StringTok{"Defense"}\NormalTok{,}\StringTok{"Sp.Attack"}\NormalTok{,}\StringTok{"Sp.Defense"}\NormalTok{,}\StringTok{"Speed"}\NormalTok{)) }\OperatorTok\StringTok{ }
\StringTok{  }\NormalTok{kableExtra}\OperatorTok{::}\KeywordTok{kable_styling}\NormalTok{()}
\end{Highlighting}
\end{Shaded}

\begin{verbatim}
## Warning in kableExtra::kable_styling(.): Please specify format in kable.
## kableExtra can customize either HTML or LaTeX outputs. See https://
## haozhu233.github.io/kableExtra/ for details.
\end{verbatim}

\begin{longtable}[]{@{}lrrrrrr@{}}
\toprule
Name & HP & Attack & Defense & Sp.Attack & Sp.Defense &
Speed\tabularnewline
\midrule
\endhead
Bulbasaur & 45 & 49 & 49 & 65 & 65 & 45\tabularnewline
Ivysaur & 60 & 62 & 63 & 80 & 80 & 60\tabularnewline
Venusaur & 80 & 82 & 83 & 100 & 100 & 80\tabularnewline
Mega Venusaur & 80 & 100 & 123 & 122 & 120 & 80\tabularnewline
Charmander & 39 & 52 & 43 & 60 & 50 & 65\tabularnewline
\bottomrule
\end{longtable}

There is hardly no two identical Pokemons,

\hypertarget{the-legendary-pokemon}{%
\subsection{\texorpdfstring{\textbf{The Legendary
Pokemon}}{The Legendary Pokemon}}\label{the-legendary-pokemon}}

Boardly, Pokemons can be divided into two groups:

\hypertarget{picture-of-fav-legendaty-poks}{%
\subparagraph{Picture of fav legendaty
Poks}\label{picture-of-fav-legendaty-poks}}

\hypertarget{plot-legendary-vs-normal-stats}{%
\subparagraph{Plot legendary vs normal
stats}\label{plot-legendary-vs-normal-stats}}

\hypertarget{plot-legendary-numbers-by-types}{%
\subparagraph{Plot legendary numbers by
types}\label{plot-legendary-numbers-by-types}}

\hypertarget{pokemon-with-multiple-forms}{%
\subparagraph{Pokemon with multiple
forms}\label{pokemon-with-multiple-forms}}


\end{document}
